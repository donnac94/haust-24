\documentclass[12pt]{article}
\usepackage[margin=0.75in]{geometry}
\geometry{a4paper}
\usepackage[T1]{fontenc} 
\usepackage[utf8]{inputenc} 
\usepackage{graphicx} 
\usepackage{hyperref} 
\usepackage{amsmath}
\usepackage{multicol}
\usepackage{tcolorbox}
\usepackage{xcolor}
\usepackage{tikz}

% Define a custom command for fill-in-the-blank
\newcommand{\fillin}[2][4cm]{%
  \begin{tikzpicture}[baseline=(current bounding box.south)]
    \draw[thick] (0,0) -- (#1,0);
    \if\relax\detokenize{#2}\relax
      % Do nothing if #2 is empty
    \else
      \node[above=2pt] at (0.5*#1,0) {#2};
    \fi
  \end{tikzpicture}%
}


\title{
\centerline{\includegraphics[width=50mm]{images/ru_logo.jpg}}
Hópverkefni 1 (e. Group Assignment 1)
\large  \\
T-117-STR1, Strjál stærðfræði I, 2024-3 \\ 
\small Reykjavik University - Department of Computer Science, Menntavegi 1, IS-101 Reykjavík, Iceland 
  }

\author{
    Kennari: Harpa Guðjónsdóttir\\
    \texttt{harpagud@ru.is}
}

\date{Skilafrestur (e. Deadline): 03.09.2024}
\begin{document}
\maketitle
\noindent
Hér er Einstaklingsverkefni 2. Skilafrestur er þriðjudaginn 10.september 2024 kl. 23:59*. Þetta eru ein af 5 einstaklingsskilum. Þau gilda alls 20\% af lokaeinkunn, en lægstu einkunn er sleppt.
\vspace{2mm}
\newline
Mjög mikilvægt er að nemendur noti þetta skjal, fylli inn sínar lausnir á viðeigandi staði og skili útfylltu skjali á Gradescope sem pdf. Bæði er leyfilegt að prenta út skjalið, fylla inn handvirkt og skanna það svo aftur inn (\textbf{eða} nota þetta \LaTeX{} sniðmát og fylla inn í það). \textbf{Ekki verður farið yfir verkefni sem ekki nota þetta skjal (eða \LaTeX{} sniðmátið), og fyrir slík verkefni fæst 0 í einkunn.}
\vspace{2mm}
\newline
\footnotesize
*nemendur á Austurlandi skila miðvikudaginn 11.september 2023 kl.23:59 og skilafrestur þeirra í Canvas/Gradescope er stilltur miðað við það
\normalsize	
\subsubsection*{English version:}
("This is the second individual assignment. The deadline is Tuesday, September 10th, 2024, at 23:59*. Students hand in solutions on pdf on Gradescope. This is one of 5 individual assignments. All in all, their weight is 20\% of the final grade, but the lowest grade is dropped.")
\vspace{2mm}
\newline
Students must use this document, fill in their solutions in the designated spaces, and return the completed document to Gradescope as a pdf. You are allowed to print the document, fill it in writing, and scan it, (\textbf{or} use this \LaTeX{} template and fill it in). \textbf{Assignments solutions that do not use this document (or the \LaTeX{} template) will not be reviewed and will receive a grade of 0.}
\vspace{2mm}
\newline
\footnotesize
*Students in the east of Iceland hand in on Wednesday, September 11th, 2024, at 23:59, and their deadline is set in Canvas/Gradescope accordingly
\normalsize	
\newpage
\section*{\textbf{Skiladæmi (e. Hand-in problems) :}}
\subsection*{Dæmi 1 (e. Problem 1) (14\%)\label{section:problem1}}
Búið til sanntöflu fyrir samsettu yrðinguna $\neg(r\land \neg q) \to(p\lor \neg r)$. Sýnið öll milliskref í töflunni, og munið að fylla inn í hausinn á töflunni. Er yrðingin sísanna? \\ \\
("Write the truth table for the proposition $\neg(r\land \neg q) \to(p\lor \neg r)$. Show all intermediate columns, and remember to fill in the head row. Is the proposition a tautology?")

\subsection*{Svar við Dæmi 1 (e. Answer to Problem 1)}

\begin{center}
    \begin{tabular}{ |c|c|c|p{0.8 cm}|p{0.8 cm}|p{1.8 cm}|p{2 cm}|p{2 cm}|p{3.8 cm}|} 
        \hline
        $p$ & $q$ & $r$ & & & & & & \\
        \hline
        T & T & T &  &  &  &  &  & \\ 
        \hline
        T & T & F &  &  &  &  &  & \\ 
        \hline
        T & F & T &  &  &  &  &  & \\ 
        \hline
        T & F & F &  &  &  &  &  & \\ 
        \hline
        F & T & T &  &  &  &  &  & \\
        \hline
        F & T & F &  &  &  &  &  & \\ 
        \hline
        F & F & T &  &  &  &  &  & \\ 
        \hline
        F & F & F &  &  &  &  &  & \\
        \hline
    \end{tabular}
\end{center}
Er yrðingin sísanna? ("Is the proposition a tautology?")
\fillin[5cm]{
    % Start your answer to a) after this line:
  
    % End your answer to a) before this line.
    }     
\newpage
\subsection*{Dæmi 2 (e. Problem 2) (20\%)\label{section:problem2}}
Sýnið að samsetta yrðingin $(\neg q\land (p\lor q)) \to p$ sé sísanna \underline{\textbf{án þess að nota sanntöfur}}. Notið umskrift á formúlum eins og gert var í fyrirlestri 2.1 Rökfræðireglur og kvantarar.  Gætið þess að nota \underline{\textbf{bara eina}} grunnreglu í hverju skrefi og vitna í hana (reglurnar eru í töflum 6 og 7 á bls 29). Sama aðferð er notuð í sýnidæmum  6, 7 og 8 í kafla 1.3 í kennslubókinni. Í kennslubókinni er leyft að beita mörgum reglum í einu skrefi, en í þessu dæmi megið þið eingöngu beita \underline{einni reglu í einu} og munið að vitna í regluna sem þið notið í hverju skrefi.\\
\\
(“Show that the conditional statement in $(\neg q\land (p\lor q)) \to p$ is a tautology \underline{\textbf{without using truth}}. \underline{\textbf{tables}}. Use the logical equivalences in tables 6 and 7 on page 29 to rewrite the formula as was done in lecture 2.1 Logic Rules and quantifiers, and as is done in Examples 6, 7 and 8 in section 1.3 of the book. In the book they sometimes use more than one rule at a time, but for this problem you must use \underline{\textbf{only one}} logical equivalence in each step and refer to it.”)

\subsection*{Svar við Dæmi 2 (e. Answer to Problem 2)}

\newcommand{\makeAnswerBoxProblemFour}{
  \noindent
  \begin{tcolorbox}[colframe=black, colback=white, boxrule=0.5pt, arc=0pt, outer arc=0pt, height=\dimexpr\textheight-\ht\strutbox-16\baselineskip\relax]
    % Start your answer after this line:

    
    % End your answer before this line.
  \end{tcolorbox}
}

\makeAnswerBoxProblemFour

\subsection*{Dæmi 3 (e. Problem 3) (16\%)\label{section:daemi3}}
Sannið seinni gleypiregluna (e.absorption law) í töflu 1 í kafla 2.2 í bókinni, þ.e. sýnið að ef $A$ og $B$ eru mengi, þá gildir að $A \cap (A \cup B) = A$. Notið mengja rithátt (e. set builder notation), skilgreiningar á snið og sammengjum\textasteriskcentered \hspace{0.5mm} og rökfræði jafngildi (töflu 6 í kafla 1) til verksins. \\
("Prove the second absorption law from Table 1 in chapter 2.2, i.e. by showing that if $A$ and $B$ are sets, then
$A \cap (A \cup B) = A$. Use set builder notation, definitions of union and intersection\textasteriskcentered \hspace{0.5mm}  and logical equivalences (table 6, chapter 1). \\ \\
\footnotesize
\textbf{Hint:} Í fyrirlestri 3.1 þá var sannað með mengja rithátt (e. set builder notation), skilgreiningu á fyllimengi og rökfræði jafngildi að $\overline{(\overline{A})} =A$. \\ 
("\textbf{Hint:} In lecture 3.1 it was proved with set builder notation, definition of complement law and logical equivalences that $\overline{(\overline{A})} =A$.")
\\ \\
\textasteriskcentered Skilgreiningar á sammengi og sniðmengi ("Definitions of union and intersection"):
$$A \cup B = \{x| x \in A \lor x \in B\}$$
$$A \cap B = \{x| x \in A \land x \in B\}$$
\normalsize

\subsection*{Svar við Dæmi 3 (e. Answer to Problem 3)}

\newcommand{\makeAnswerBoxProblemThree}{
  \noindent
  \begin{tcolorbox}[colframe=black, colback=white, boxrule=0.5pt, arc=0pt, outer arc=0pt, height=\dimexpr\textheight-\ht\strutbox-18\baselineskip\relax]
    % Start your answer after this line:

    
    % End your answer before this line.
  \end{tcolorbox}
}

\makeAnswerBoxProblemThree 





\newpage
\subsection*{Dæmi 4 (e. Problem 4) (14\%+10\%) \label{section:daemi4}}
Könnun meðal 500 sjónvarpsáhorfenda gefur eftirfarandi upplýsingar. Af þeim horfa 285 á fótbolta, 195 á handbolta og 115 á körfubolta. Það horfa 70 á fótbolta og handbolta, 45 á fótbolta og körfubolta og 42 á handbolta og körfubolta. Það eru 50 sem horfa ekki á neina af þessum íþróttagreinum. \\
("A survey of 500 television viewers provides the following information. Of them 285 watch soccer, 195 watch handball and 115 watch basketball. Also, 70 watch soccer and handball, 45 watch soccer and basketball and 42 watch in handball and basketball. There are 50 viewers who do not watch any of these sports.")
\begin{itemize}
    \item[a)] Hve margir þátttakendur í könnuninni horfa á allar íþróttagreinarnar? ("ow manu watch all three sports?")
    \item[b)] Hversu margir horfa á fótbolta en hvorki handbolta né körfubolta? ("How many watch soccer but neither handball nor basketball?")
\end{itemize}
Sýnið alla útreikninga í svarkassanum, og setjið að auki lokasvörin á línurnar fyrir neðan. ("Show all calculations in the answer box, and additionally put your final answers on the lines below.")

\subsection*{Svör við Dæmi 4 (e. Answers to Problem 4)}

\newcommand{\makeAnswerBoxProblemFour}{
  \noindent
  \begin{tcolorbox}[colframe=black, colback=white, boxrule=0.5pt, arc=0pt, outer arc=0pt, height=\dimexpr\textheight-\ht\strutbox-25\baselineskip\relax]
    % Start your answer after this line:

    
    % End your answer before this line.
  \end{tcolorbox}
}

\makeAnswerBoxProblemFour 

\vspace{5mm}

\begin{multicols}{2}
\begin{itemize}
    \item [a)] \fillin[7cm]{
    % Start your final answer to a) after this line:
  
    % End your final answer to a) before this line.
    }     
    \newline
    \item [b)] \fillin[7cm]{
    % Start your final answer to b) after this line:
  
    % End your final answer to b) before this line.
    } 
    \newline   
\end{itemize}
\end{multicols}
\subsection*{Dæmi 5 (e. Problem 5) (20\%) \label{section:daemi5}}
Gefin er í kennslubókinni tafla með mengjareglum, tafla 1 í kafla 2.2. Notið mengjareglurnar í þessari töflu til að sýna fram á eftirfarandi formúlu. Vísið í eina mengjareglu í hverju einasta skrefi.
\\ \\
("There is a table with set identities in the textbook, table 1 in chapter 2.2. Use the set identities in that table to show the following. Show clearly which set identity you are using in each step")

$$B \cup \overline{(\overline{A} \cap B)} = U $$

\subsection*{Svar við Dæmi 5 (e. Answer to Problem 5)}

\newcommand{\makeAnswerBoxProblemFive}{
  \noindent
  \begin{tcolorbox}[colframe=black, colback=white, boxrule=0.5pt, arc=0pt, outer arc=0pt, height=\dimexpr\textheight-\ht\strutbox-14\baselineskip\relax]
    % Start your answer after this line:

    
    % End your answer before this line.
  \end{tcolorbox}
}

\makeAnswerBoxProblemFive
\newpage
\subsection*{Dæmi 6 (e. Problem 6) (4\%+4\%+6\%) \label{section:daemi6}}
Endurskrifið eftirfarandi fullyrðingar þannig að neitunin birtist aðeins á yrðingarfalli (en ekki fyrir utan kvantara eða sviga). Takið einungis eitt skref í einu, og þegar þið notið rökfræðireglur vísið í þær. \\ \\ 
("Rewrite each of these statements so that negations appear only within predicates (that is, so that no negation is outside a quantifier or an expression involving logical connectives). Take only one step at a time, and when you use the logical equivalence rules, refer to them.")
\begin{itemize}
    \item[a)] $\neg \forall y \forall x L(x, y)$
    \item[b)] $\neg \exists x \forall y L(x, y)$
    \item[c)] $\neg \exists y (M (y) \lor \forall x \neg L(x, y))$ 
\end{itemize}
\subsection*{Svör við Dæmi 6 (e. Answers to Problem 6)}

\newcommand{\makeAnswerBoxProblemSixA}{
  \begin{tcolorbox}[colframe=black, colback=white, boxrule=0.5pt, arc=0pt, outer arc=0pt, height=3.5cm, width=\linewidth, top=1mm, bottom=1mm, left=1mm, right=1mm]
    % Start your answer to a) after this line:

    
    % End your answer to a) before this line.
  \end{tcolorbox}
}

\newcommand{\makeAnswerBoxProblemSixB}{
  \begin{tcolorbox}[colframe=black, colback=white, boxrule=0.5pt, arc=0pt, outer arc=0pt, height=3.5cm, width=\linewidth, top=1mm, bottom=1mm, left=1mm, right=1mm]
    % Start your answer to b) after this line:

    
    % End your answer to b) before this line.
  \end{tcolorbox}
}

\newcommand{\makeAnswerBoxProblemSixC}{
  \begin{tcolorbox}[colframe=black, colback=white, boxrule=0.5pt, arc=0pt, outer arc=0pt, height=3.5cm, width=\linewidth, top=1mm, bottom=1mm, left=1mm, right=1mm]
    % Start your answer to c) after this line:

    
    % End your answer to c) before this line.
  \end{tcolorbox}
}

\begin{itemize}
    \item[a)] \makeAnswerBoxProblemSixA
    \item[b)] \makeAnswerBoxProblemSixB
    \item[c)] \makeAnswerBoxProblemSixC
\end{itemize}
\end{document}
