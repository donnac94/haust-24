\newpage
\subsection*{Dæmi 5 (e. Problem 5) (4\%+4\%+4\%+4\%+4\%) \label{section:daemi5}}
Látum $I(x)$ vera yrðinguna “$x$ er með nettengingu” og $C(x, y)$ vera yrðinguna “$x$ og $y$ hafa talað saman á netinu,” þar sem mengi breytanna $x$ og $y$ eru allir nemendur í bekknum þínum. Ritið eftirfarandi staðhæfingar með því að nota eftir því sem við á allsherjarkvantara, tilvistarkvantara, rökfræðileg tákn og föllin $I(x)$ og $C(x,y)$\\
\\
("Let $I(x)$ be the statement “$x$ has an Internet connection” and $C(x, y)$ be the statement “$x$ and $y$ have chatted over the Internet,” where the domain for the variables $x$ and $y$ consists of all students in your class. Write the statements using these predicates, universal quantification, existential quantification, logical operators as well a $I(x)$ and $C(x,y)$.

\begin{itemize}
    \item[a)] Jón er ekki með nettengingu. ("Jón does not have an Internet connection.")
    \item[b)] Rúna hefur ekki talað við Sessu á netinu. ("Rúna has not chatted over the Internet with Sessa.")
    \item[c)] Enginn í bekknum hefur talað við Bóas á netinu. ("No one in the class has chatted with Bóas over the internet.")
    \item[d)] Allir nema einn nemandi í bekknum þínum er með nettengingu. ("Everyone except one student in your class has an Internet connection.")
    \item[e)] Það eru að minnsta kosti tveir nemendur í bekknum sem hafa ekki talað við sömu manneskjuna í bekknum á netinu.
    ("There are at least two students in your class who have not chatted over the internet with the same person in your class.")
\end{itemize}

\subsection*{Svör við Dæmi 5 (e. Answers to Problem 5)}

\newcommand{\makeAnswerBoxProblemFiveA}{
  \begin{tcolorbox}[colframe=black, colback=white, boxrule=0.5pt, arc=0pt, outer arc=0pt, height=1.8cm, width=\linewidth, top=1mm, bottom=1mm, left=1mm, right=1mm]
    % Start your answer to a) after this line:

    % End your answer to a) before this line.
  \end{tcolorbox}
}

\newcommand{\makeAnswerBoxProblemFiveB}{
  \begin{tcolorbox}[colframe=black, colback=white, boxrule=0.5pt, arc=0pt, outer arc=0pt, height=1.8cm, width=\linewidth, top=1mm, bottom=1mm, left=1mm, right=1mm]
    % Start your answer to b) after this line:

    % End your answer to b) before this line.
  \end{tcolorbox}
}

\newcommand{\makeAnswerBoxProblemFiveC}{
  \begin{tcolorbox}[colframe=black, colback=white, boxrule=0.5pt, arc=0pt, outer arc=0pt, height=1.8cm, width=\linewidth, top=1mm, bottom=1mm, left=1mm, right=1mm]
    % Start your answer to c) after this line:

    % End your answer to c) before this line.
  \end{tcolorbox}
}

\newcommand{\makeAnswerBoxProblemFiveD}{
  \begin{tcolorbox}[colframe=black, colback=white, boxrule=0.5pt, arc=0pt, outer arc=0pt, height=1.8cm, width=\linewidth, top=1mm, bottom=1mm, left=1mm, right=1mm]
    % Start your answer to d) after this line:

    % End your answer to d) before this line.
  \end{tcolorbox}
}

\newcommand{\makeAnswerBoxProblemFiveE}{
  \begin{tcolorbox}[colframe=black, colback=white, boxrule=0.5pt, arc=0pt, outer arc=0pt, height=1.8cm, width=\linewidth, top=1mm, bottom=1mm, left=1mm, right=1mm]
    % Start your answer to e) after this line:

    % End your answer to e) before this line.
  \end{tcolorbox}
}

\begin{itemize}
    \item[a)] \makeAnswerBoxProblemFiveA
    \item[b)] \makeAnswerBoxProblemFiveB
    \item[c)] \makeAnswerBoxProblemFiveC
    \item[d)] \makeAnswerBoxProblemFiveD
    \item[e)] \makeAnswerBoxProblemFiveE
\end{itemize}