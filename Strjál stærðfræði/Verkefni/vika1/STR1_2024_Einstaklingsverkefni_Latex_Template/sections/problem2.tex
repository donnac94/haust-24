\newpage
\subsection*{Dæmi 2 (e. Problem 2) (4\%+4\%+4\%+4\%+4\%+4\%)\label{section:problem2}}
Látum $p$, $q$, og $r$ vera yrðingarnar ("Let $p$, $q$, and $r$ be the propositions")
\begin{itemize}
    \item [$p$ :] Þú færð 10 á lokaprófinu. ("You get 10 on the final exam.")
    \item [$q$ :] Þú leysir öll dæmin í bókinni. ("You do every exercises in the book.") 
    \item [$r$ :] Þú færð 10 í áfanganum. ("You get 10 in this class.") 
\end{itemize}
Skrifið þessar yrðingar niður með því að nota $p$, $q$, og $r$ og rökfræðivirkja (líka neitun). ("Write these propositions using $p$, $q$, and $r$ and logical connectives (including negations).")

\begin{itemize}
    \item [a)] Þú færð 10 í áfanganum, en þú leystir ekki öll dæmin í bókinni. \\("You get a 10 in this class, but you do not do every exercise in the book.")
     \item [b)] Þú færð 10 á lokaprófinu, þú leysir öll dæmin í bókinni, og þú færð 10 í áfanganum.\\ ("You get a 10 on the final exam, you do every exercise in the book, and you get a 10 in this class.")
     \item [c)] Til þess að fá 10 í áfanganum, þá er nauðsynlegt fyrir þig að fá 10 á lokaprófinu.\\ ("To get a 10 in this class, it is necessary for you to get a 10 on the final exam.")
     \item [d)] Þú færð 10 á lokaprófinu, en þú leystir ekki öll dæmin í bókinni; engu að síður færð þú 10 í áfanganum.\\ ("You get a 10 on the final exam, but you don´t do every exercise in the book; nevertheless, you get a 10 in this class.")
     \item [e)] Að fá 10 á lokaprófinu og leysa sérhvert dæmi í bókinni er nægjanlegt til að fá 10 í þessum áfanga.\\("Getting a 10 on the final exam and doing every exercise in the book is sufficient for getting a 10 in this class.")
     \item [f)] Þú færð 10 í áfanganum þá og því aðeins að þú annað hvort leysir öll dæmin í bókinni eða þú færð 10 á lokaprófinu.\\ ("You will get a 10 in this class if and only if you either do every exercise in the book or you get a 10 on the final exam.")
\end{itemize}
\subsection*{Svör við Dæmi 2 (e. Answers to Problem 2)}
\vspace{0.5cm}
\begin{multicols}{2}
\begin{itemize}
    \item [a)] \fillin[7cm]{
    % Start your answer to a) after this line:
  
    % End your answer to a) before this line.
    }     
    \newline
    \item [b)] \fillin[7cm]{
    % Start your answer to b) after this line:
  
    % End your answer to b) before this line.
    } 
    \newline
    \item [c)] \fillin[7cm]{
    % Start your answer to c) after this line:
  
    % End your answer to c) before this line.
    }  
    \newline
    \item [d)] \fillin[7cm]{
    % Start your answer to d) after this line:
    
    % End your answer to d) before this line.
    }  
    \newline
    \item [e)] \fillin[7cm]{
    % Start your answer to e) after this line:
  
    % End your answer to e) before this line.
    }  
    \newline
    \item [f)] \fillin[7cm]{
    % Start your answer to f) after this line:
  
    % End your answer to f) before this line.
    }  
    \newline   
\end{itemize}
\end{multicols}
