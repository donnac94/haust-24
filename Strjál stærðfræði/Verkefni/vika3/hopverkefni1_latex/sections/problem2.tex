\newpage
\subsection*{Dæmi 2 (e. Problem 2) (20\%)\label{section:problem2}}
Sýnið að samsetta yrðingin $(\neg q\land (p\lor q)) \to p$ sé sísanna \underline{\textbf{án þess að nota sanntöfur}}. Notið umskrift á formúlum eins og gert var í fyrirlestri 2.1 Rökfræðireglur og kvantarar.  Gætið þess að nota \underline{\textbf{bara eina}} grunnreglu í hverju skrefi og vitna í hana (reglurnar eru í töflum 6 og 7 á bls 29). Sama aðferð er notuð í sýnidæmum  6, 7 og 8 í kafla 1.3 í kennslubókinni. Í kennslubókinni er leyft að beita mörgum reglum í einu skrefi, en í þessu dæmi megið þið eingöngu beita \underline{einni reglu í einu} og munið að vitna í regluna sem þið notið í hverju skrefi.\\
\\
(“Show that the conditional statement in $(\neg q\land (p\lor q)) \to p$ is a tautology \underline{\textbf{without using truth}}. \underline{\textbf{tables}}. Use the logical equivalences in tables 6 and 7 on page 29 to rewrite the formula as was done in lecture 2.1 Logic Rules and quantifiers, and as is done in Examples 6, 7 and 8 in section 1.3 of the book. In the book they sometimes use more than one rule at a time, but for this problem you must use \underline{\textbf{only one}} logical equivalence in each step and refer to it.”)

\subsection*{Svar við Dæmi 2 (e. Answer to Problem 2)}

\newcommand{\makeAnswerBoxProblemFour}{
  \noindent
  \begin{tcolorbox}[colframe=black, colback=white, boxrule=0.5pt, arc=0pt, outer arc=0pt, height=\dimexpr\textheight-\ht\strutbox-16\baselineskip\relax]
    % Start your answer after this line:

    
    % End your answer before this line.
  \end{tcolorbox}
}

\makeAnswerBoxProblemFour
