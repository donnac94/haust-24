\newpage
\subsection*{Dæmi 4 (e. Problem 4) (4\%+4\%+4\%+4\%) \label{section:daemi4}}
Gefnar eru eftirfarandi opnar yrðingar: ("We have the following statements:")
\begin{itemize}
    \item[] $I(x)$: einstaklingurinn $x$ hefur staðist inntökupróf. ("the person $x$ has passed an entrance exam.")
    \item[] $M(x)$ einstaklingurinn $x$ er í tónlistarskóla. ("the person x studies at a music school.")
    \item[] $L(x,y)$ einstaklingurinn $x$ hefur lokið námskeiðinu $y$. ("the person $x$ has completed the course $y$.")
\end{itemize}
Ritið eftirfarandi staðhæfingar með því að nota eftir því sem við á allsherjarkvantara, tilvistarkvantara, rökfræðileg tákn og föllin $I(x)$, $M(x)$ og $L(x,y)$. ("Write the statements using these predicates, universal quantification, existential quantification, logical operators as well as $I(x)$, $M(x)$ and $L(x,y)$.")
\begin{itemize}
    \item[a)] Anna er í tónlistarskóla en Finnur er ekki í tónlistarskóla. ("Anna studies at a music school but Finnur does not study at a music school.")
    \item[b)] Allir sem eru í tónlistarskóla hafa staðist inntökupróf. ("Everyone who studies at a music school has passed an entrance exam.")
\end{itemize}
Skrifið á mæltu máli eftirfarandi yrðingar: ("Express each of these by an English sentence:") 
\begin{itemize}
    \item[c)] $\forall x \exists y (L(x,y))$
    \item[d] $\exists y\forall x (\neg L(x,y))$
\end{itemize}

\subsection*{Svör við Dæmi 4 (e. Answers to Problem 4)}

\newcommand{\makeAnswerBoxProblemFourA}{
  \begin{tcolorbox}[colframe=black, colback=white, boxrule=0.5pt, arc=0pt, outer arc=0pt, height=2.2cm, width=\linewidth, top=1mm, bottom=1mm, left=1mm, right=1mm]
    % Start your answer to a) after this line:

    
    % End your answer to a) before this line.
  \end{tcolorbox}
}

\newcommand{\makeAnswerBoxProblemFourB}{
  \begin{tcolorbox}[colframe=black, colback=white, boxrule=0.5pt, arc=0pt, outer arc=0pt, height=2.2cm, width=\linewidth, top=1mm, bottom=1mm, left=1mm, right=1mm]
    % Start your answer to b) after this line:

    
    % End your answer to b) before this line.
  \end{tcolorbox}
}

\newcommand{\makeAnswerBoxProblemFourC}{
  \begin{tcolorbox}[colframe=black, colback=white, boxrule=0.5pt, arc=0pt, outer arc=0pt, height=2.2cm, width=\linewidth, top=1mm, bottom=1mm, left=1mm, right=1mm]
    % Start your answer to c) after this line:

    
    % End your answer to c) before this line.
  \end{tcolorbox}
}

\newcommand{\makeAnswerBoxProblemFourD}{
  \begin{tcolorbox}[colframe=black, colback=white, boxrule=0.5pt, arc=0pt, outer arc=0pt, height=2.2cm, width=\linewidth, top=1mm, bottom=1mm, left=1mm, right=1mm]
    % Start your answer to d) after this line:

    
    % End your answer to d) before this line.
  \end{tcolorbox}
}

\begin{itemize}
    \item[a)] \makeAnswerBoxProblemFourA
    \item[b)] \makeAnswerBoxProblemFourB
    \item[c)] \makeAnswerBoxProblemFourC
    \item[d)] \makeAnswerBoxProblemFourD    
\end{itemize}