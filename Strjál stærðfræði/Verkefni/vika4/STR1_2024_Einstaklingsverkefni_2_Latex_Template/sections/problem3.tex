\subsection*{Dæmi 3 (e. Problem 3) (16\%)\label{section:daemi3}}
Sannið seinni gleypiregluna (e.absorption law) í töflu 1 í kafla 2.2 í bókinni, þ.e. sýnið að ef $A$ og $B$ eru mengi, þá gildir að $A \cap (A \cup B) = A$. Notið mengja rithátt (e. set builder notation), skilgreiningar á snið og sammengjum\textasteriskcentered \hspace{0.5mm} og rökfræði jafngildi (töflu 6 í kafla 1) til verksins. \\
("Prove the second absorption law from Table 1 in chapter 2.2, i.e. by showing that if $A$ and $B$ are sets, then
$A \cap (A \cup B) = A$. Use set builder notation, definitions of union and intersection\textasteriskcentered \hspace{0.5mm}  and logical equivalences (table 6, chapter 1). \\ \\
\footnotesize
\textbf{Hint:} Í fyrirlestri 3.1 þá var sannað með mengja rithátt (e. set builder notation), skilgreiningu á fyllimengi og rökfræði jafngildi að $\overline{(\overline{A})} =A$. \\ 
("\textbf{Hint:} In lecture 3.1 it was proved with set builder notation, definition of complement law and logical equivalences that $\overline{(\overline{A})} =A$.")
\\ \\
\textasteriskcentered Skilgreiningar á sammengi og sniðmengi ("Definitions of union and intersection"):
$$A \cup B = \{x| x \in A \lor x \in B\}$$
$$A \cap B = \{x| x \in A \land x \in B\}$$
\normalsize

\subsection*{Svar við Dæmi 3 (e. Answer to Problem 3)}

\newcommand{\makeAnswerBoxProblemThree}{
  \noindent
  \begin{tcolorbox}[colframe=black, colback=white, boxrule=0.5pt, arc=0pt, outer arc=0pt, height=\dimexpr\textheight-\ht\strutbox-18\baselineskip\relax]
    % Start your answer after this line:

    
    % End your answer before this line.
  \end{tcolorbox}
}

\makeAnswerBoxProblemThree 




